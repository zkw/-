\documentclass[content.tex]{subfiles}

\title{小学数学思维潜力课 / 四年级暑假 \\ (时间问题)}

\begin{document}

\begin{frame}
\maketitle
\end{frame}

\begin{frame}{时间问题}
\begin{itemize}
\item $1$ 分钟有 $60$ 秒.
\item $1$ 小时有 $60$ 分钟, $3600$ 秒.
\item $1$ 天有 $24$ 小时, $86400$ 秒.
\item 
下列月份有 $31$ 天: 1, 3, 5, 7, 8, 10, 12 \\
下列月份有 $30$ 天: 4, 6, 9, 11 \\
闰年 2 月有 $29$ 天, 平年 2 月有 $28$ 天. \\
\item 四年一闰, 百年不闰, 四百年又闰.
\end{itemize}
\bigbreak
\centering (例 1,2,3,4: 简单题, 不讲.) \\
\end{frame}

\begin{frame}{例 5}
建建 $30$ 秒钟能写 $8$ 个字, $1$ 分钟能写多少个字? \\
$9$ 分钟能写多少个字? 
\begin{exampleblock}{题解}
$1$ 分钟是两个 $30$ 秒, 能写 $16$ 个字. \\
$9$ 分钟能写 $16 \times 9 = 144$ 个字.
\end{exampleblock}
\end{frame}

\begin{frame}{例 6}
王教授出国讲学, \\
5 月 3 日零点出国, 8 月 5 日早晨 $8$ 点回国. \\
问王教授在国外的时间有多长? 
\begin{exampleblock}{题解}
5 月份有 $31$ 天, 去掉前两天, 在国外 $29$ 天. \\
6 月份有 $30$ 天, 在国外. \\
7 月份有 $31$ 天, 在国外. \\
8 月份前 $4$ 天和第五天的前 $8$ 小时, 在国外. \\
总共 $94$ 天零 $8$ 个小时.
\end{exampleblock}
\end{frame}

\begin{frame}{习题 A1}
如果现在的时间是 $9$ 时 $35$ 分, 再过 $40$ 分是几点几分?
\begin{exampleblock}{题解}
再过 $40$ 分是 $9$ 时 $75$ 分, 即 $10$ 时 $15$ 分.
\end{exampleblock}
\end{frame}

\begin{frame}{习题 A2}
现在的时间是上午 $10$ 时, $30$ 分钟以前是几点几分?
\begin{exampleblock}{题解}
借一当 $60$, 得到 $9$ 时 $30$ 分.
\end{exampleblock}
\end{frame}

\begin{frame}{习题 A3}
小林妈妈从家到单位要用 $40$ 分钟, 早上 $8:00$ 上班, \\ 她最晚几点几分就要离家?
\begin{exampleblock}{题解}
借一当 $60$, 得到 $7$ 时 $20$ 分.
\end{exampleblock}
\end{frame}

\begin{frame}{习题 A4}
小明 $7$ 时 $40$ 分去上学, \\
他提前 $1$ 小时起床, 提前 $30$ 分钟吃早饭, \\
小明几点几分起床, 几点几分开始吃早饭? \\
\begin{exampleblock}{题解}
做减法, 小明 $6$ 点 $40$ 分起床, $7$ 点 $10$ 分开始吃早饭. \\
\end{exampleblock}
\end{frame}

\begin{frame}{习题 A5}
小学生每天睡 $9$ 小时, 小红早上 $6:30$ 起床, \\
她应该在几点几分上床睡觉. 
\begin{exampleblock}{题解}
做减法, 借一当 $24$. 答案是 $21:30$. \\
\end{exampleblock}
\end{frame}

\begin{frame}{习题 A6}
从早晨 $7$ 时到下午 $3$ 时, 经过了几小时? 
\begin{exampleblock}{题解}
做减法, 用 $24$ 小时制. $15 - 7 = 8$. 
\end{exampleblock}
\end{frame}

\begin{frame}{习题 A7}
一家百货商店早晨 $8$ 时开始营业, 晚上 $8$ 时停止营业, \\ 这家商店每天营业几小时?
\begin{exampleblock}{题解}
做减法, 用 $24$ 小时制. $20 - 8 = 12$.
\end{exampleblock}
\end{frame}

\begin{frame}{习题 A8}
$1$ 个人跳一支舞的时间是 $8$ 分钟, \\
$5$ 个人同时跳一支舞的时间是几分钟?
\begin{exampleblock}{题解}
注意是同时, 因此也是 $8$ 分钟.
\end{exampleblock}
\bigbreak
\centering (习题 A9, A10: 排版较难, 不讲.) \\
\end{frame}

\begin{frame}{习题 B1}
一辆汽车早晨 $5$ 时从起点站出发, $7$ 时 $20$ 分到达终点站, \\
路上用了几时几分?
\begin{exampleblock}{题解}
做减法, $2$ 时 $20$ 分.
\end{exampleblock}
\end{frame}

\begin{frame}{习题 B2}
世贸大厦上午 $8$ 时 $30$ 分开始营业, \\
下午 $6$ 时 $30$ 分停止营业, 每天营业多少小时? 
\begin{exampleblock}{题解}
使用 $24$ 小时制, 做减法, $18 - 8 = 10$ 小时.
\end{exampleblock}
\end{frame}

\begin{frame}{习题 B3}
同学们参加夏令营活动, 早上 $6$ 时出发, \\
晚上 $6$ 时 $10$ 分回来, 共用了几时几分?
\begin{exampleblock}{题解}
使用 $24$ 小时制, 做减法, $18:10 - 6:00 = 12:10$ .
\end{exampleblock}
\end{frame}

\begin{frame}{习题 B4}
一节课是 $40$ 分钟, 休息 $10$ 分钟后再上下一节课. \\
从早上 $8:30$ 上第一节课, 第二节课何时下课? 
\begin{exampleblock}{题解}
$40+10+40=90$ 分钟, $8:30 + 90 = 10:00$ .
\end{exampleblock}
\end{frame}

\begin{frame}{习题 B5}
跳跳做作业, \\
完成数学作业用了规定时间的一半, \\
完成语文作业又用了剩下时间的一半, \\
最后用 9 分钟完成读书作业, 跳跳完成全部作业用了多少时间? 
\begin{exampleblock}{题解}
设总时间是 $x$ 分钟, 则完成数学作业用了 $x/2$ 分钟.
$(x/2)+(x/4)+9 = x \implies x = 36$ .
\end{exampleblock}
\end{frame}

\begin{frame}{习题 B6}
从中午 $12$ 时到午夜零时, 时针和分针重叠多少次?
\begin{exampleblock}{题解}
中午 $12$ 时重叠第一次, $1$ 时大约 $5$ 分时候重叠第二次. \\
以此类推, $1 \sim 2$, \quad $2 \sim 3$, \quad $\ldots$, \quad $11\sim 12$ 
共重叠 $11$ 次, \\
连上头尾两次, 共 $13$ 次.
\end{exampleblock}
\end{frame}

\begin{frame}{习题 B7}
一只台钟逢半点敲 $1$ 下, 正点为几时就敲几下, \\
那么这只台钟从 $1$ 点到 $12$ 点共敲几下? \\
\begin{exampleblock}{题解}
半点共敲 $12$ 下, 整点共敲 
$$1+2+3+\cdots+12=\frac{13\times 12}{2}=78\;.$$
共 $12+78=90$ 下.
\end{exampleblock}
\end{frame}

\begin{frame}{习题 B8}
有一个闹钟一昼夜快 $3$ 分钟, 小华要赶火车, \\
想让这个钟在明天早上北京时间 $8$ 点准时闹, \\
那么当闹钟走到今天下午 $4$ 点时, 应该往慢拨多少分钟?
\begin{exampleblock}{题解}
今天 $16$ 时到明天 $8$ 时, 是 $16$ 小时. \\
$3\div 24\times 16=2$ 分钟, 因此会差 $2$ 分钟,应该拨 $2$ 分钟.
\end{exampleblock}
\end{frame}

\begin{frame}{习题 B9}
王芳每天学习 $6$ 小时, 锻炼身体 $1$ 小时 $30$ 分钟, \\
文娱活动 $2$ 小时 $30$ 分钟, 用餐 $2$ 小时, 睡眠 $9$ 小时. \\
问王芳每天还有多少时间自由活动?
\begin{exampleblock}{题解}
$24-6-1.5-2.5-2-9=3$ 小时.
\end{exampleblock}
\end{frame}

\begin{frame}{习题 B10}
每年的第三季度 (7 -- 9 月) 和第四季度 (10 -- 12 月) 各有多少天?
\begin{exampleblock}{题解}
第三季度: $31+31+30 = 92$ 天. \\
第四季度: $31+30+31 = 92$ 天. \\
\end{exampleblock}
\end{frame}

\end{document}
