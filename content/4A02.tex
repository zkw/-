\documentclass[content.tex]{subfiles}

\title{小学数学思维潜力课 / 四年级暑假 \\ (速算与巧算)}

\begin{document}

\begin{frame}
\maketitle
\end{frame}

\begin{frame}{知识概要}
为了使运算简便, 可以采用以下方法.
\begin{itemize}
\item 凑整: 凑成整十整百的数.
\item 改变运算顺序: 利用加法交换律和结合律.
\item 添括号: 如果括号前是加号不变号, 否则变号.
\end{itemize}
\end{frame}

\begin{frame}{例 1}
求值: $1+2+3+4+5+6+7+8+9$
\begin{exampleblock}{题解}
$\text{原式}=(1+9)+(2+8)+(3+7)+(4+6)+5=45$.
\end{exampleblock}
\end{frame}

\begin{frame}{例 2}
求值: $3+5+6+8+25+32+44+77$
\begin{exampleblock}{题解}
根据个位, 两两配对.
\begin{align*}
\text{原式}&=(3+77)+(5+25)+(6+44)+(8+32)\\
&=80+30+50+40=200\;.
\end{align*}
\end{exampleblock}
\end{frame}

\begin{frame}{例 3}
求值: $10-9+8-7+6-5+4-3+2-1$
\begin{exampleblock}{题解}
添括号.
\begin{align*}
\text{原式} &= (10-9)+(8-7)+(6-5)+(4-3)+(2-1)\\
&=1+1+1+1+1=5\;.
\end{align*}
\end{exampleblock}
\end{frame}

\begin{frame}{例 4}
求值: $58+44+6$
\begin{exampleblock}{题解}
添括号. $\text{原式} = 58 + (44 + 6) = 58 + 50 = 108$ .
\end{exampleblock}
\end{frame}

\begin{frame}{例 5}
求值: $490+440-140$
\begin{exampleblock}{题解}
添括号. $\text{原式} = 490 + (440 - 140) = 490+300 = 790$ .
\end{exampleblock}
\end{frame}

\begin{frame}{例 6}
求值: $72-140+28+540$
\begin{exampleblock}{题解}
交换次序, 添括号. 
\begin{align*}
\text{原式} &= 72+28+540-140 \\
&= (72+28) + (540-140) = 100 + 400 = 500\;.
\end{align*}
\end{exampleblock}
\end{frame}

\begin{frame}{例 7}
求值: $49+24$
\begin{exampleblock}{题解}
前面的数加 $1$, 后面的数减 $1$, 和不变. \\
$\text{原式} = 50 + 23 = 73$ .
\end{exampleblock}
\bigbreak
\centering (例 8: 直接算即可, 跳过不讲.)
\end{frame}

\begin{frame}{例 9}
求值: $876-99$
\begin{exampleblock}{题解}
前面的数和后面的同时加 $1$, 结果不变. \\
$\text{原式} = 877 - 100 = 777$ .
\end{exampleblock}
\end{frame}

\begin{frame}{例 10}
求值: $53+49+18$
\begin{exampleblock}{题解}
叠加一个 $-3+1+2=0$ 和不变. \\
$\text{原式} = 50+50+20 = 120$ .
\end{exampleblock}
\end{frame}

\begin{frame}{例 11}
求值: $8+18+28+38$
\begin{exampleblock}{题解}
叠加一个 $-6+2+2+2=0$ 和不变. \\
$\text{原式} = 2+20+30+40 = 92$ .
\end{exampleblock}
\end{frame}

\begin{frame}{例 12}
求值: $2998+995+99+48$
\begin{exampleblock}{题解}
叠加一个 $2+5+1+2-10=0$ 和不变.
\begin{align*}
\text{原式} &= 3000+1000+100+50-10 \\
&= (3000+1000+100) + (50-10) = 4100 + 40 = 4140\;.
\end{align*}
\end{exampleblock}
\end{frame}

\begin{frame}{习题 A1}
求值: $1+3+5+7+9+11+13+15+17+19$
\begin{exampleblock}{题解}
$\text{原式}=(1+19)+(3+17)+(5+15)+(7+13)+(9+11)=100$.
\end{exampleblock}
\bigbreak
\centering (习题 A2: 同例 8, 跳过不讲.)
\end{frame}

\begin{frame}{习题 A3}
求值: $5+79+35$
\begin{exampleblock}{题解}
交换运算次序. $\text{原式} = 5+35+79 = 40+79 = 119$ .
\end{exampleblock}
\end{frame}

\begin{frame}{习题 A4}
求值: $40+370+30$
\begin{exampleblock}{题解}
添括号. $\text{原式} = 40+(370+30) = 40+400 = 440$ .
\end{exampleblock}
\end{frame}

\begin{frame}{习题 A5}
求值: $2+4+6+8+10+12+14+16+18+20$
\begin{exampleblock}{题解}
配对凑整.
\begin{align*}
\text{原式} &= (2+18)+(4+16)+(6+14)+(8+12)+10+20 \\
&= 20+20+20+20+10+20=110\;.
\end{align*}
\end{exampleblock}
\end{frame}

\begin{frame}{习题 A6}
求值: $1+3+21+27+39+59$
\begin{exampleblock}{题解}
叠加一个 $-1-3-1+3+1+1=0$ 和不变.
$$\text{原式} = 0+0+20+30+40+60 =150\;.$$
\end{exampleblock}
\end{frame}

\begin{frame}{习题 A7}
求值: $21+22+23+24+25+26+27+28+29$
\begin{exampleblock}{题解}
等差数列, 平均数等于中间数. $\text{原式} = 25\times 9 = 225$ .
\end{exampleblock}
\end{frame}

\begin{frame}{习题 A8}
求值: $25+27+29+31+33+35$
\begin{exampleblock}{题解}
等差数列, 平均数等于中间数. $\text{原式} = 30\times 6 = 180$ .
\end{exampleblock}
\end{frame}

\begin{frame}{习题 A9}
求值: $44-40+36-32+28-24$
\begin{exampleblock}{题解}
添括号.
\begin{align*}
\text{原式} &= (44-40)+(36-32)+(28-24)\\
&=4+4+4=12\;.
\end{align*}
\end{exampleblock}
\end{frame}

\begin{frame}{习题 A10}
求值: $(22+24+26+28+30)-(21+23+25+27+29)$
\begin{exampleblock}{题解}
交换运算次序. 
\begin{align*}
\text{原式} &= (22-21)+(24-23)+(26-25)+(28-27)+(30-29)\\
&=1+1+1+1+1=5\;.
\end{align*}
\end{exampleblock}
\end{frame}

\begin{frame}{习题 A11}
求值: $16+58$
\begin{exampleblock}{题解}
前面的数减 $2$, 后面的数加 $2$, 和不变. \\
$\text{原式} = 14 + 60 = 74$ .
\end{exampleblock}
\end{frame}

\begin{frame}{习题 A12}
求值: $565+99$
\begin{exampleblock}{题解}
前面的数减 $1$, 后面的数加 $1$, 和不变. \\
$\text{原式} = 564 + 100 = 664$ .
\end{exampleblock}
\bigbreak
\centering (习题 A13: 直接算即可, 跳过不讲.)
\end{frame}

\begin{frame}{习题 A14}
求值: $28 + 45$
\begin{exampleblock}{题解}
前面的数加 $2$, 后面的数减 $2$, 和不变. \\
$\text{原式} = 30 + 43 = 73$ .
\end{exampleblock}
\end{frame}

\begin{frame}{习题 A15}
求值: $44+69$
\begin{exampleblock}{题解}
前面的数减 $1$, 后面的数加 $1$, 和不变. \\
$\text{原式} = 43 + 70 = 113$ .
\end{exampleblock}
\end{frame}

\begin{frame}{习题 A16}
求值: $47+35$
\begin{exampleblock}{题解}
前面的数加 $3$, 后面的数减 $3$, 和不变. \\
$\text{原式} = 50 + 32 = 82$ .
\end{exampleblock}
\end{frame}

\begin{frame}{习题 A17}
求值: $345+97$
\begin{exampleblock}{题解}
前面的数减 $3$, 后面的数加 $3$, 和不变. \\
$\text{原式} = 342 + 100 = 442$ .
\end{exampleblock}
\end{frame}

\begin{frame}{习题 A18}
求值: $104+898$
\begin{exampleblock}{题解}
前面的数减 $2$, 后面的数加 $2$, 和不变. \\
$\text{原式} = 102+900 = 1002$ .
\end{exampleblock}
\bigbreak
\centering (习题 A19: 直接算即可, 跳过不讲.)
\end{frame}

\begin{frame}{习题 A20}
求值: $393-198$
\begin{exampleblock}{题解}
前面的数加 $2$, 后面的数加 $2$, 和不变. \\
$\text{原式} = 395-200 = 195$ .
\end{exampleblock}
\end{frame}

\begin{frame}{习题 B1}
求值: $510+270+290$
\begin{exampleblock}{题解}
交换运算次序. $\text{原式} = 510+290+270 = 800+270 = 1070$ .
\end{exampleblock}
\end{frame}

\begin{frame}{习题 B2}
求值: $490+440-140$
\begin{exampleblock}{题解}
添括号. $\text{原式} = 490+(440-140) = 490+300 = 790$ .
\end{exampleblock}
\end{frame}

\begin{frame}{习题 B3}
求值: $430-50+170$
\begin{exampleblock}{题解}
交换运算次序. $\text{原式} = 430+170-50 = 600-50 = 550$ .
\end{exampleblock}
\end{frame}

\begin{frame}{习题 B4}
求值: $45+370+55-70$
\begin{exampleblock}{题解}
交换次序, 添括号. 
\begin{align*}
\text{原式} &= 45+55+370-70 \\
&= (45+55) + (370-70) = 100 + 300 = 400\;.
\end{align*}
\end{exampleblock}
\end{frame}

\begin{frame}{习题 B5}
求值: $15+46+4$
\begin{exampleblock}{题解}
添括号. $\text{原式} = 15+(46+4) = 15+50 = 65$ .
\end{exampleblock}
\end{frame}

\begin{frame}{习题 B6}
求值: $65+43+25$
\begin{exampleblock}{题解}
交换运算次序. $\text{原式} = 65+25+43 = 90+43 = 133$ .
\end{exampleblock}
\end{frame}

\begin{frame}{习题 B7}
求值: $9+41+8+32$
\begin{exampleblock}{题解}
添括号. $\text{原式} = (9+41)+(8+32) = 50+40 = 90$ .
\end{exampleblock}
\end{frame}

\begin{frame}{习题 B8}
求值: $21+33+17+9$
\begin{exampleblock}{题解}
交换次序, 添括号. 
\begin{align*}
\text{原式} &= 21+9+33+17 \\
&= (21+9) + (33+17) = 30+50=80\;.
\end{align*}
\end{exampleblock}
\end{frame}

\begin{frame}{习题 B9}
求值: $97+260+140$
\begin{exampleblock}{题解}
添括号. $\text{原式} = 97+(260+140) = 97+400 = 497$ .
\end{exampleblock}
\end{frame}

\begin{frame}{习题 B10}
求值: $430+78+170$
\begin{exampleblock}{题解}
交换运算次序. $\text{原式} = 430+170+78 = 600+78 = 678$ .
\end{exampleblock}
\end{frame}

\begin{frame}{习题 B11}
求值: $15+16+17+4+5$
\begin{exampleblock}{题解}
交换运算次序. $\text{原式} = (15+5)+(16+4)+17 = 20+20+17=57$ .
\end{exampleblock}
\end{frame}

\begin{frame}{习题 B12}
求值: $6998+995+97$
\begin{exampleblock}{题解}
叠加一个 $2-5+3=0$ 和不变.
$$\text{原式} = 7000+990+100 =8090\;.$$
\end{exampleblock}
\end{frame}

\begin{frame}{习题 B13}
求值: $999+97+68+6$
\begin{exampleblock}{题解}
叠加一个 $1+3+2-6=0$ 和不变.
$$\text{原式} = 1000+100+70 = 1170\;.$$
\end{exampleblock}
\end{frame}

\begin{frame}{习题 B14}
求值: $84+29+37$
\begin{exampleblock}{题解}
叠加一个 $-4+1+3=0$ 和不变.
$$\text{原式} = 80+30+40=150\;.$$
\end{exampleblock}
\end{frame}

\begin{frame}{习题 B15}
求值: $28+39+33$
\begin{exampleblock}{题解}
叠加一个 $2+1-3=0$ 和不变.
$$\text{原式} = 30+40+30=100\;.$$
\end{exampleblock}
\end{frame}

\begin{frame}{习题 B16}
求值: $18+38+24$
\begin{exampleblock}{题解}
叠加一个 $2+2-4=0$ 和不变.
$$\text{原式} = 20+40+20=80\;.$$
\end{exampleblock}
\end{frame}

\begin{frame}{习题 B17}
求值: $995+98+9$
\begin{exampleblock}{题解}
叠加一个 $5+2+7=0$ 和不变.
$$\text{原式} = 1000+100+2=1102\;.$$
\end{exampleblock}
\end{frame}

\begin{frame}{习题 B18}
求值: $3999+397+89$
\begin{exampleblock}{题解}
叠加一个 $1+3+1-5=0$ 和不变.
$$\text{原式} = 4000+400+90-5=4485\;.$$
\end{exampleblock}
\end{frame}

\begin{frame}{习题 B19}
求值: $4+9+99+999+2999$
\begin{exampleblock}{题解}
叠加一个 $-4+1+1+1+1=0$ 和不变.
$$\text{原式} = 10+100+1000+3000=4110\;.$$
\end{exampleblock}
\end{frame}

\begin{frame}{习题 B20}
求值:
\begin{itemize}
\item $1+2+3+ \cdots +10$
\item $1+2+3+ \cdots +100$
\item $1+2+3+ \cdots +1000$
\item $1+2+3+ \cdots +10000$
\end{itemize}
\begin{exampleblock}{题解}
$$\text{等差数列的和} = 
\frac{(\text{首项} + \text{末项})\times \text{项数}}{2}\;.$$ 
答案是 $55, 5050, 500500, 50005000$.
\end{exampleblock}
\end{frame}
\end{document}
