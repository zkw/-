\documentclass[content.tex]{subfiles}

\title{小学数学思维潜力课 / 四年级暑假 \\ (等式的加减法)}

\begin{document}

\begin{frame}
\maketitle
\end{frame}

\begin{frame}{知识概要}
\begin{itemize}
\item 代入: 把式子中的某部分用相等的量替换.
\item 加减: 等式两边分别相加减, 仍是等式.
\item 消元: 减少未知数的数量.
\end{itemize}
\end{frame}

\begin{frame}{例 1}
两个数的和等于$20$, 差等于$2$, 求这两个数.
\begin{exampleblock}{题解}
把两个式子左右分别加起来即可.
$$
\begin{cases}
A+B=20\\
A-B=2\\
\end{cases}
\implies 2A=22
\implies A=11\;.
$$
\end{exampleblock}
\end{frame}

\begin{frame}{例 2}
已知 $A + B = 15$, $A - B = 9$. 求 $A,B$ .
\begin{exampleblock}{题解}
把两个式子左右分别相减, 得到
$$2B = 6 \implies B = 3\;.$$
代入第二个式子, 得到
$$A = B + 9 = 12\;.$$
\end{exampleblock}
\textbf{举一反三: 一开始能否把两个式子相加?}
\end{frame}

\begin{frame}{例 3}
已知 $A + B = 36$, $B = A + A + A$. 求 $A,B$ .
\begin{exampleblock}{题解}
把第二个式子代入第一个式子, 得到
$$
4A = 36 \implies A = 9
$$
代入第二个式子, 得到 $B = 27$ . 
\end{exampleblock}
\end{frame}

\begin{frame}{例 4}
已知 $2A+B=22$, $A+2B=20$. 求 $A,B$ .
\begin{exampleblock}{题解}
把第二个式子变为 $A=20-2B$ 代入第一个式子, 得到
$$2(20-2B)+B=22\implies 40-3B=22\implies B=6\;.$$
代入, 得到 $A=20-2\times6=8$ . 
\end{exampleblock}
\textbf{举一反三: 能否把第二个等式乘以 $2$, 然后两等式相减? }
\end{frame}

\begin{frame}{例 5}
已知 $2A+3B=27$, $2A+5B=35$. 求 $B$ .
\begin{exampleblock}{题解}
两等式相减, 得到 $2B = 8\implies B = 4$ .
\end{exampleblock}
\end{frame}

\begin{frame}{例 6}
君君买 $1$ 支铅笔和 $2$ 块橡皮共用去 $4$ 角 $6$ 分, \\
$1$ 支铅笔比 $2$ 块橡皮贵 $1$ 角 $4$ 分, 求每只铅笔的价格.
\begin{exampleblock}{题解}
设一支铅笔 $A$ 分, 一块橡皮 $B$ 分. \\
$A+2B=46$, $A-2B=14$, 求和得 $2A=60\implies A=30$ .
\end{exampleblock}
\end{frame}

\begin{frame}{例 7}
在一次数学竞赛中, \\
丁丁和成成的成绩加起来是 $193$ 分, \\
丁丁和秋秋的成绩加起来是 $196$ 分, \\
成成和秋秋得成绩加起来是 $191$ 分, \\
问他们三人各得多少分?
\begin{exampleblock}{题解}
设丁丁成成秋秋分别为 $A,B,C$ 分.
$$A+B=193,\quad A+C=196,\quad B+C=191\;.$$
为了求出 $A$, 变化成 $B = 193 - A, C = 196 - A$, 代入得
$$(193-A)+ (196-A)=191$$
$\implies 389-2A=191 \implies A = 99$ .
\end{exampleblock}
\textbf{举一反三: 一开始能否把三个式子相加?}
\end{frame}

\begin{frame}{例 8}
古生物科学家从化石中发现了两种已经灭绝了的古代动物, \\
每种头上长有一定数目的角. \\
$5$ 只 A 种动物与 $2$ 只 B 种动物共有 $65$ 个角, \\
$3$ 只 A 种动物与 $4$ 只 B 种动物共有 $81$ 个角, \\
那么 A,B 两种动物每只各有多少个角? 
\begin{exampleblock}{题解}
$$5A+2B=65,\quad 3A+4B=81.$$
将第一个式子两边乘以 $2$, 得 $10A+4B=130$. \\
和第二个式子相减, 
$$7A=49\implies A=7\;.$$
代入得 $B=(65-5\times7)\div 2=15$ .
\end{exampleblock}
\end{frame}

\begin{frame}{例 9}
一本科技书的页数比一本故事书的页数多 11 页, \\
如果 2 本这样的科技书与 3 本这样的故事书合起来有 972 页, \\
问两种书每本各多少页?
\begin{exampleblock}{题解}
设故事书为 $x$ 页, 则科技书 $x+11$ 页.
$$2(x+11)+3x=972\implies 5x+22=972 \implies x=190\;.$$
\end{exampleblock}
\end{frame}

\begin{frame}{例 10}
两个整数分别用 $A, B$ 表示. $2A+7B=90$, $4A+9B=140$, \\ 求两数之和与两数之差.
\begin{exampleblock}{题解}
第一个式子乘以 $2$ 和第二个式子相减, 得 $5B=40\implies B=8$.\\
代入求出 $A=17$, 并求出 $A+B=25, A-B=9.$
\end{exampleblock}
\end{frame}

\begin{frame}{习题 A1}
$\text{鸡}+\text{鸭}=300$, $\text{鸡}-\text{鸭}=100$. 求鸡和鸭.
\begin{exampleblock}{题解}
两式相加, $2\text{鸡} = 400\implies \text{鸡} = 200$. \\
代入得, $\text{鸭} = \text{鸡} - 100 = 100$.
\end{exampleblock}
\end{frame}

\begin{frame}{习题 A2}
$2A+B=56$, $2A+2B=72$. 求 $A, B$ .
\begin{exampleblock}{题解}
两式相减, $B=16$. 代入得, $A=(56-16)/2=20$.
\end{exampleblock}
\end{frame}

\begin{frame}{习题 A3}
$A+B=10$, $3A+2B=26$. 求 $A, B$ .
\begin{exampleblock}{题解}
$$
\begin{cases}
A+B=10 & (1) \\
3A+2B=26 & (2) \\
\end{cases}
$$
$(2) - (1) \times 2$, $A=6$. 代入得, $B=10-6=4$.
\end{exampleblock}
\end{frame}

\begin{frame}{习题 A4}
$2A+B=13$, $3A+2B=21$. 求 $A, B$ .
\begin{exampleblock}{题解}
$$
\begin{cases}
2A+B=13 & (1) \\
3A+2B=21 & (2) \\
\end{cases}
$$
$(1) \times 2 - (2)$, $A=5$. 代入得, $B=13 - 5\times 2=3$.
\end{exampleblock}
\end{frame}

\begin{frame}{习题 A5}
$A+B=10$, $A-B=2$. 求 $A, B$ .
\begin{exampleblock}{题解}
两式相加, $2A=12\implies A=6$. 代入得, $B = 6 - 2 = 4$.
\end{exampleblock}
\end{frame}

\begin{frame}{习题 A6}
$A+B+A+A=27$, $B+B=A+A+A$. 求 $A, B$ .
\begin{exampleblock}{题解}
$$
\begin{cases}
3A+B=27 & (1) \\
2B=3A & (2) \\
\end{cases}
$$
$(2)$ 代入 $(1)$, $3B=27\implies B=9$. \\
代入得, $A=2B/3=6$.
\end{exampleblock}
\end{frame}

\begin{frame}{习题 A7}
$41-A-B=5$, $A=B+B+B$. 求 $A, B$ .
\begin{exampleblock}{题解}
代入得, $41-4B=5\implies B=9$. $A=3B=27$.
\end{exampleblock}
\end{frame}

\begin{frame}{习题 A8.1}
$3A=21$, $2A+B=18$. 求 $A-B$.
\begin{exampleblock}{题解}
$A=7$. 代入得 $B=4$. $\implies A-B=3$.
\end{exampleblock}
\end{frame}

\begin{frame}{习题 A8.2}
$A-B=38$, $A-C=45$. 求 $B-C$.
\begin{exampleblock}{题解}
这题3个未知数2个方程, 一般是求不出具体值.\\
第二个式子减第一个式子, $B-C=7$.
\end{exampleblock}
\end{frame}

\begin{frame}{习题 A8.3}
$3A=15$, $3B=12$, 求 $A+B$.
\begin{exampleblock}{题解}
$A=5$, $B=4$. $\implies A + B = 9$.
\end{exampleblock}
\end{frame}

\begin{frame}{习题 A9}
参加夏令营活动的学生共有 $96$ 人, 男生比女生多 $8$ 人, \\ 算一算男、女生各有多少人?
\begin{exampleblock}{题解}
设女生 $x$ 人, 则男生 $x + 8$ 人.
$$x + (x + 8) = 96\implies x=44\;.$$
\end{exampleblock}
\end{frame}

\begin{frame}{习题 A10}
已知排球加篮球是 $62$ 个, 排球减篮球是 $12$ 个, \\
问排球、篮球各多少个?
\begin{exampleblock}{题解}
设排球 $x$ 个, 篮球 $y$ 个. $x + y = 62,\quad x - y = 12$ . \\
两式相加, $2x=74\implies x = 37$ . 代入得 $y = 25$ .
\end{exampleblock}
\end{frame}

\begin{frame}{习题 A11}
$6$ 筐香蕉加 $6$ 筐苹果等于 $390$ 千克, \\
$1$ 筐香蕉减 $1$ 筐苹果等于 $5$ 千克, \\
$1$ 筐香蕉几千克? $1$ 筐苹果几千克?
\begin{exampleblock}{题解}
$1$ 筐香蕉加 $1$ 筐苹果等于 $65$ 千克, \\
和第二个式子相加, $2$ 筐香蕉 $70$ 千克, 一筐香蕉 $35$ 千克.\\
一筐苹果 $65-35=30$ 千克.
\end{exampleblock}
\end{frame}

\begin{frame}{习题 A12}
小明、小刚和小虎三人共有课外书 $49$ 本, 小明比小刚多 $4$ 本, 小刚又比小虎多 $6$ 本, 问三人各有多少本?
\begin{exampleblock}{题解}
设小虎有 $x$ 本, 则小刚 $x + 6$ 本, 小明 $x + 10$ 本.
$$x+(x+6)+(x+10) = 49\implies 3x=33\implies x = 11\;.$$
\end{exampleblock}
\end{frame}

\begin{frame}{习题 B1}
$
\begin{cases}
\text{梨}+\text{苹果} = 38\\
\text{苹果}-\text{梨} = 14\\
\end{cases}
$, 求苹果和梨? 
\begin{exampleblock}{题解}
两式相减, $2\text{梨}=24\implies \text{梨} = 12$ . \\
$\text{苹果} = \text{梨} + 14 = 26$ .
\end{exampleblock}
\end{frame}

\begin{frame}{习题 B2}
$
\begin{cases}
A + B = 9\\
A - B = 1\\
\end{cases}
$, 求 $A, B$? 
\begin{exampleblock}{题解}
两式相减, $2B=8\implies B=4$ . \\
$A = B + 1 = 5$ .
\end{exampleblock}
\end{frame}

\begin{frame}{习题 B3}
$
\begin{cases}
A + B = 16\\
B = 3A\\
\end{cases}
$, 求 $A, B$? 
\begin{exampleblock}{题解}
代入, $4A = 16\implies A=4$ . \\
$B = 3A = 12$ .
\end{exampleblock}
\end{frame}

\begin{frame}{习题 B4}
$A + B = A - B$, 你能得出什么结论?
\begin{exampleblock}{题解}
把右边的项移到左边,得 $2B=0\implies B = 0$ .
对 $A$ 没有限制.
\end{exampleblock}
\end{frame}

\begin{frame}{习题 B5}
$
\begin{cases}
2A + 2B = 24\\
A + 2B = 14\\
\end{cases}
$, 求 $A, B$? 
\begin{exampleblock}{题解}
相减, $A = 10$ . $B = (14-A)/2 = 2$ .
\end{exampleblock}
\end{frame}

\begin{frame}{习题 B6}
$
\begin{cases}
2A + B + C = 16\\
A + 2B + C = 13\\
A + B + 2C = 11\\
\end{cases}
$, 求 $A, B, C$? 
\begin{exampleblock}{题解}
三式相加, $4A + 4B + 4C = 40\implies A + B + C = 10$ . \\
分别用各式减去此式, $A = 6$, $B = 3$, $C = 1$.
\end{exampleblock}
\end{frame}

\begin{frame}{习题 B7}
$
\begin{cases}
A + 20 + B = 25\\
B = A - 3\\
\end{cases}
$, 求 $A, B$? 
\begin{exampleblock}{题解}
代入, $A + 20 + (A - 3) = 25\implies A = 4$ . \\
代入得, $B = 1$.
\end{exampleblock}
\end{frame}

\begin{frame}{习题 B8}
二年级两个班做纸花, 一共做了 $56$ 朵, \\
二 (1) 班比二 (2) 班多做 $6$ 朵, \\
二 (1) 班和二 (2) 班各做了多少朵纸花?
\begin{exampleblock}{题解}
设二 (2) 班做了 $x$ 朵, 则二 (1) 班做了 $x + 6$ 朵. \\
$x + (x + 6) = 56\implies x = 25$ .
\end{exampleblock}
\end{frame}

\begin{frame}{习题 B9}
二年级三个班分本子, \\
二 (1) 班、二 (2) 班共分到 $129$ 本, \\
二 (2) 班、二 (3) 班共分到 $103$ 本, \\
二 (1) 班、二 (3) 班共分到 $118$ 本, \\
各分到多少本?
\begin{exampleblock}{题解}
设各分到 $a_1, a_2, a_3$ 本.
$
\begin{cases}
a_1 + a_2 = 129\\
a_2 + a_3 = 103\\
a_1 + a_3 = 118\\
\end{cases}
$, 求和得到 
$2a_1 + 2a_2 + 2a_3 = 350\implies a_1 + a_2 + a_3 = 175$ . \\
分别减去三个式子, $a_1 = 72$, $a_2 = 57$, $a_3 = 46$.
\end{exampleblock}
\end{frame}

\begin{frame}{习题 B10}
$3$ 只布熊的价钱加 $4$ 个布娃娃的价钱等于 $62$ 元, \\
$3$ 只布熊的价钱加 $1$ 个布娃娃的价钱等于 $38$ 元, \\
问一只布熊多少钱, 一个布娃娃多少钱?
\begin{exampleblock}{题解}
相减, $3$ 个布娃娃 $24$ 元, 布娃娃 $8$ 元. \\
代入, 布熊 $(38-8)/3 = 10$ 元. 
\end{exampleblock}
\end{frame}

\begin{frame}{习题 B11}
书架上层:$6$ 本科技书, $3$ 本故事书\\
书架下层:$4$ 本科技书, $4$ 本故事书\\
书架两层总定价相等, 已知科技书每本 $6$ 元, 故事书每本多少钱?
\begin{exampleblock}{题解}
设故事书每本 $x$ 元. $6\times 6 + 3x = 4\times 6 + 4x$. \\
移项, $x = 36-24 = 12$ 元.
\end{exampleblock}
\end{frame}

\begin{frame}{习题 B12}
有白红黑三色的球. \\
白的和红的合在一起有 $10$ 个, \\
红的和黑的合在一起有 $7$ 个, \\
黑的和白的合在一起有 $5$ 个, \\
问三种球合在一起共多少个?
\begin{exampleblock}{题解}
把三个式子加起来, $2\text{白} + 2\text{红} + 2\text{黑}=22$. \\
故一共 11 个.
\end{exampleblock}
\end{frame}

\end{document}
