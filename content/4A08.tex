\documentclass[content.tex]{subfiles}

\title{小学数学思维潜力课 / 四年级暑假 \\ (简单的周期问题)}

\begin{document}

\begin{frame}
\maketitle
\end{frame}

\begin{frame}{例 1}
八个自然数排成一排, \\
从第三个数开始, 每个数都是前面两个数之和. \\
已知第五个数是 $7$, 求第八个数.
\begin{exampleblock}{题解}
如何用尽可能少的未知数表示出这八个数呢? \\
设前两个数是 $a,b$, 则这些数分别是 
$$
\begin{matrix}
a, & b, & a+b, & a+2b, &  \\
2a+3b, & 3a+5b, & 5a+8b, & 8a+13b, \\
\end{matrix}
$$
其中 $2a+3b=7$. 枚举得 $b = 1, a = 2$ , \\
代入得第八个数 $8a+13b=29$. 
\end{exampleblock}
\end{frame}

\begin{frame}{例 2}
从 $1$ 开始连续 $n$ 个自然数的和的个位可以有多少种不同的数字?
\begin{exampleblock}{题解}
\begin{tabular}{c|cccccccccccccccccccc}
自然数 $n$ & 1 & 2 & 3 & 4 & 5 & 6 & 7 & 8 & 9 & 10 \\
\footnotesize $1$ 到 $n$ 的和个位 & 
1 & 3 & 6 & 0 & 5 & 1 & 8 & 6 & 5 & 0 \\
\hline
自然数 $n$ & 11 & 12 & 13 & 14 & 15 & 16 & 17 & 18 & 19 & 20 \\
\footnotesize $1$ 到 $n$ 的和个位 & 
6 & 8 & 1 & 5 & 0 & 6 & 3 & 1 & 0 & 0 \\
\end{tabular}
\bigbreak
前 $20$ 个和的个位见表格, 从 $21$ 开始循环. \\
因此有 $6$ 种不同的数字 $0,1,3,5,6,8$.
\end{exampleblock}
\end{frame}

\begin{frame}{例 3}
A, B, C, D, E 五个盒子中依次放有 $9,5,3,2,1$ 个小球. \\
第 $1$ 个小朋友找到放球最少的盒子, \\
然后从其他盒子中各取一个球放入这个盒子. \\
第 $2$ 个小朋友找到放球最少的盒子, \\
然后从其他盒子中各取一个球放入这个盒子... \\
当 $1000$ 位小朋友放完后, 五个盒子中各放有几个球?
\begin{exampleblock}{题解}
$$
\begin{matrix}
    & (9,5,3,2,1) & \to & (8,4,2,1,5) & \to & (7,3,1,5,4) \\
\hline
\to & (6,2,5,4,3) & \to & (5,6,4,3,2) & \to & (4,5,3,2,6) \\
\to & (3,4,2,6,5) & \to & (2,3,6,5,4) \\
\hline
\to & (6,2,5,4,3) & \to & \cdots \\
\end{matrix}
$$
第 1000 个小朋友后面, 应该是第 1001 组数. \\
去掉前面的 3 组, 是 998 组. $998\div 5 = 199 \text{余} 3$, \\
因此情况应该和循环中的第 $3$ 组一样, 即 $(4,5,3,2,6)$ .
\end{exampleblock}
\end{frame}

\begin{frame}{例 4}
有红黄蓝三种颜色的彩旗 $160$ 面, \\
按 $4$ 面红旗, $3$ 面黄旗, $2$ 面蓝旗的顺序排列挂着, \\
那么最后一面彩旗是什么颜色? 红旗共有几面?
\begin{exampleblock}{题解}
一个周期内有 $4+3+2=9$ 面旗. $160\div 9=17\text{余}7$. \\
最后的 $7$ 面旗子是 $4$ 红 $3$ 黄, 因此最后一面旗子是黄色的. \\
红旗共有 $17\times 4 + 4 = 72$ 面.
\end{exampleblock}
\end{frame}

\begin{frame}{例 5}
2003 年的儿童节是星期日, 那么 2008 年的国庆节是星期几?
\begin{exampleblock}{题解}
2003 年 6 -- 12 月, 天数为 
$$30+31+31+30+31+30+31=214\;.$$
2004 -- 2007 年, 天数为
$$366+365+365+365=1461\;.$$
2008 年 1 -- 9 月, 天数为 
$$31+29+31+30+31+30+31+31+30=274\;.$$
算头不算尾, 经过的天数为 $214+1461+274=1949$ . \\
$1949\div 7=278\text{余}3$, 从星期日向后三天, 是星期三.
\end{exampleblock}
\end{frame}

\begin{frame}{例 6}
把自然数按下表规律排列后, 可分成 A, B, C, D, E 五类, \\
例如 4 在 E 类, 11 在 D 类, 那么 $997$ 在哪一行的哪一类?
\bigbreak\centering
\begin{tabular}{c|cccccc}
\hline
A && 8 && 16 && $\cdots$ \\
B & 1 & 7 & 9 & 15 & 17 & $\cdots$ \\
C & 2 & 6 & 10 & 14 & $\cdots$ & $\cdots$ \\
D & 3 & 5 & 11 & 13 & $\cdots$ & $\cdots$ \\
E & 4 && 12 && $\cdots$ \\
\hline
\end{tabular}
\begin{exampleblock}{题解}
每列 $4$ 个数, $997\div 4 = 249\text{余}1$, 
因此是第 $250$ 列的第一个数. \\
又因为奇数列从左向右, 偶数列从右向左, \\
所以是从右向左第一个数, 是 D 类.
\end{exampleblock}
\end{frame}

\begin{frame}{例 7}
下面表格的字母, 汉字和数字组成一组, \\
第一组是 $(A, \text{上}, 2)$, \\
第二组是 $(B, \text{海}, 0)$, \\
那么第 $19$ 组的字母, 汉字和数字是什么? \\
第 $30$ 组的字母, 汉字和数字又是什么呢? \\
\begin{exampleblock}{题解}
字母以 $3$ 为周期. \\
$19\div 3 = 6\text{余}1$, 第 $19$ 个字母是 A. \\
$30\div 3 = 10\text{余}0$, 第 $30$ 个字母是 C. \\
汉字以 $5$ 为周期. \\
$19\div 5 = 3\text{余}4$, 第 $19$ 个汉字是 $\text{博}$. \\
$30\div 5 = 6\text{余}0$, 第 $30$ 个汉字是 $\text{会}$. \\
数字以 $4$ 为周期. \\
$19\div 4 = 4\text{余}3$, 第 $19$ 个数字是 $1$. \\
$30\div 4 = 7\text{余}2$, 第 $30$ 个数字是 $0$. \\
\end{exampleblock}
\end{frame}

\begin{frame}[fragile]{例 8}
\begin{columns}
\column{.7\textwidth}
如右图, 电子跳蚤每跳一步, \\
可从一个圆圈跳到相邻的圆圈. \\
现在, 一只红跳蚤从标有数 $1$ 的圆圈 \\
按顺时针方向跳了 $50$ 步, 落在一个圆圈里. \\
一只黑跳蚤也从标有数 $7$ 的圆圈 \\
按逆时针方向跳了 $60$ 步, 落在另一个圆圈里. \\
这两个圆圈里数的乘积是多少?
\column{.3\textwidth}
\begin{asy}[width=.8\textwidth]
draw(unitcircle);
for (int i = 1; i < 8; ++i) {
    pair pos = expi(pi / 2.0 - 2.0 * pi / 7.0 * (i - 1));
    filldraw(circle(pos, 0.2), yellow, black);
    label(format("$%d$", i), pos);
    // $
}
\end{asy}
\end{columns}
\begin{exampleblock}{题解}
红跳蚤相当于跳了 $1$ 步, 黑跳蚤相当于跳了 $4$ 步. \\
红跳蚤跳到了 $2$ 的位置, 黑跳蚤跳到了 $3$ 的位置. \\
其积为 $2 \times 3 = 6$.
\end{exampleblock}
\end{frame}

\begin{frame}{习题 A1}
有 $38$ 颗糖, 按小光, 小明, 小红, 小华四个人的顺序, \\
每人每次分到一颗糖, 温水分到最后一颗糖.
\begin{exampleblock}{题解}
$38\div 4 = 9\text{余}2$, 因此小明分到最后一颗糖.
\end{exampleblock}
\end{frame}

\begin{frame}{习题 A2}
有一堆棋子, 按照下面顺序排列起来, 第 $36$ 棋子是什么颜色? 
$$
\text{\ding{108}}\text{\ding{108}}
\text{\ding{109}}\text{\ding{109}}\text{\ding{109}}
\text{\ding{108}}\text{\ding{108}}
\text{\ding{109}}\text{\ding{109}}\text{\ding{109}}
\text{\ding{108}}\text{\ding{108}}
\text{\ding{109}}\text{\ding{109}}\text{\ding{109}}
\text{\ding{108}}\text{\ding{108}}
\text{\ding{109}}\text{\ding{109}}\text{\ding{109}}
$$
\begin{exampleblock}{题解}
$36\div 5 = 7\text{余}1$, 因此是黑色.
\end{exampleblock}
\end{frame}

\begin{frame}{习题 A3}
老师有 1 -- 50 号卡片依次发给小红, 小林, 冬冬和兰兰四个人, \\
第 $38$ 张卡片应该发给谁?
\begin{exampleblock}{题解}
$38\div 4 = 9\text{余}2$, 因此是小林.
\end{exampleblock}
\end{frame}

\begin{frame}{习题 A4}
一列数按 $385161613851616138516161\cdots$ 规律排列, \\
第 $71$ 个数是多少?
\begin{exampleblock}{题解}
$71\div 8 = 8\text{余}7$, 因此是 $6$.
\end{exampleblock}
\end{frame}

\begin{frame}{习题 A5}
把一列数 $1,3,5,7, \ldots$ 按下图这样排列: \\
\begin{center}
\begin{tabular}{ccccc}
第 $1$ 组 & 第 $2$ 组 & 第 $3$ 组 & 第 $4$ 组 & 第 $5$ 组 \\
$1$ & $3$ & $5$ & $7$ & $9$ \\
$11$ & $13$ & $15$ & $17$ & $19$ \\
$21$ & $23$ & $\cdots$ \\
\end{tabular}
\end{center}
那么 93 这个数应该排在第几组?
\begin{exampleblock}{题解}
$93\div 10 = 9\text{余}3$, 因此在第二组.
\end{exampleblock}
\end{frame}

\begin{frame}{习题 A6}
小明和小丽玩摸牌的游戏, \\
按照小明摸 $3$ 张, 小丽摸 $4$ 张的规则. \\
一副牌 $54$ 张, 谁摸到最后一张牌?
\begin{exampleblock}{题解}
$54\div 7 = 7\text{余}5$, 因此小丽摸到最后一张.
\end{exampleblock}
\end{frame}

\begin{frame}{习题 A7}
课外活动中, 甲乙丙丁四人排成一个圆圈依次报数. \\
甲报 $1$, 乙报 $2$, 丙报 $3$, 丁报 $4$, \\
这样每人报的数总比前一个人多 $1$, \\
则 $27$ 是谁报的? $54$ 又是谁报的?
\begin{exampleblock}{题解}
$27\div 4 = 6\text{余}3$, 因此丙报的 $27$. \\
$54\div 4 = 6\text{余}2$, 因此乙报的 $54$. \\
\end{exampleblock}
\end{frame}

\begin{frame}{习题 A8}
2002 年 3 月 1 日是星期五, 3 月 31 日是星期几?
\begin{exampleblock}{题解}
从 1 日到 31 日经过了 30 天, 相当于经过两天. \\
因此是星期日.
\end{exampleblock}
\end{frame}


\end{document}
