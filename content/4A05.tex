\documentclass[content.tex]{subfiles}

\title{小学数学思维潜力课 / 四年级暑假 \\ (横式谜)}

\begin{document}

\begin{frame}
\maketitle
\end{frame}

\begin{frame}{例 1}
将 $1 \sim 8$ 这八个数字分别填入 $\square$, 使等式成立.
$$
\square\square\square\square -
\square\square\square\square = 1111
$$
\begin{exampleblock}{题解}
差是 $1$, 除非 $10-9=1$, 否则没借位. \\
但只有 $1 \sim 8$, 故确实没借位. \\
把 $1 \sim 8$ 分成四组两两差 $1$, 只有一种方法, 就是
$$(1,2),(3,4),(5,6),(7,8)\;,$$
排列得 $24$ 种答案.
\end{exampleblock}
\end{frame}

\begin{frame}{例 2}
将 $1 \sim 9$ 分别填入下列两个算式的 $\square$ 内, \\
使每个等式都成立 (其中三个数字已填好).
$$
\begin{cases}
\square \times \square = 5\square \\
12 + \square = \square + \square \\
\end{cases}
$$
\begin{exampleblock}{题解}
枚举第一个式子右边, $54\text{或}56$. \\
若是 $54$, $6 \times 9 = 54$, 剩下 $3, 7, 8$, 有 $12+3=7+8$. \\
若是 $56$, $7 \times 8 = 56$, 剩下 $3, 4, 9$, 不合题意.
\end{exampleblock}
\end{frame}

\begin{frame}{例 3}
将 $1 \sim 9$ 分别填入下列两个算式的 $\square$ 内, \\
使每个等式都成立 (其中两个数字已填好).
$$
\begin{cases}
\square + \square - \square  = \square \\
\square \times \square \div \square = 16 \\
\end{cases}
$$
\begin{exampleblock}{题解}
枚举第二个式子除数, $2\text{或}3\text{或}4$. \\
若是 $2$, 得 $4 \times 8 \div 2 = 16$, 剩下有 $5+7=3+9$, 可填入. \\
若是 $3$, 得 $6 \times 8 \div 3 = 16$, 重复了 $6$. \\
若是 $4$, 得 $8 \times 8 \div 4 = 16$, 重复了 $8$. \\
\end{exampleblock}
\end{frame}

\begin{frame}{例 4}
将 $0 \sim 9$ 分别填入下列两个算式的 $\square$ 内, \\
使每个等式都成立.
$$
\begin{cases}
\square + \square = \square \\
\square - \square = \square \\
\square \times \square = \square \square \\
\end{cases}
$$
\begin{exampleblock}{题解}
首先数字 $0$ 肯定在第二个式子积的个位. \\
枚举积的十位, 可以是 $1, 2, 3, 4$. \\
若是 $1$, 得 $2 \times 5 = 10$, 剩下 $3,4,6,7,8,9$ 不合题意. \\
若是 $2$, 得 $4 \times 5 = 20$, 剩下 $1+7=8,\quad 9-3=6$. \\
若是 $3$, 得 $6 \times 5 = 30$, 剩下 $1,2,4,7,8,9$ 不合题意. \\
若是 $4$, 得 $8 \times 5 = 40$, 剩下 $1,2,3,6,7,9$ 不合题意. \\
\end{exampleblock}
\end{frame}

\begin{frame}{例 5}
将 $0 \sim 9$ 分别填入下列算式的 $\square$ 内, 
使每个等式都成立.
\begin{align*}
5\times(\square - 8) &= 5 &
\square \div 2 + 3 &= 6 &
\square \times \square + 3 &= 27 \\
(\square + 2) \div 6 &= \square &
2 \times \square + \square &= 10 &
2 \times (\square + \square) &= 10
\end{align*}
\begin{exampleblock}{题解}
第一个式子填 $9$, 第二个式子填 $6$. \\
第三个式子积是 $24 = 3 \times 8$. \\
第四个式子商只能是 $1$, 被加数是 $4$. \\
还剩 $0,2,5,7$. 凑一下, 第五个式子是 $2\times5+0=10$, \\
第六个式子是 $2\times(7 - 2) = 10$.
\end{exampleblock}
\end{frame}

\begin{frame}{例 6}
万事如意这四个汉字分别表示一个 $10$ 以内的偶数, \\
请将下列四个算式用正确的数字表达出来.
$$
\begin{cases}
\text{万} - \text{事} \div \text{如} + \text{意} = 9 & (1)\\
\text{万} - \text{事} \div \text{如} - \text{意} = 1 & (2)\\
\text{意} - \text{事} \div \text{如} - \text{万} = 9 & (3) \\
\text{意} - (\text{万} - \text{事}) \div \text{如} = 3 & (4) \\
\end{cases}
$$
\begin{exampleblock}{题解}
$[(1) - (2)] / 2 \implies \text{意} = 4$. 
$\text{如}$ 是除数, 只能是 $2$. \\
剩下 $6, 8$ 可以试验一下.
\end{exampleblock}
\end{frame}

\begin{frame}{习题 A1}
\begin{align*}
A\times 9 &= 81&\implies A&=\_\_\_\_. \\
B+B+B&=21&\implies B&=\_\_\_\_. \\
6\times C - 3 &= 15&\implies C &= \_\_\_\_.
\end{align*}
\begin{exampleblock}{题解}
一般的解方程方法就可以了.
\end{exampleblock}
\end{frame}

\begin{frame}{习题 A2}
将 $1 \sim 9$ 分别填入下列算式的 $\square$ 内, 
使每个等式都成立.
\begin{align*}
\square + \square &= \square \\
\square - \square &= \square \\
\square \times \square &= \square
\end{align*}
\begin{exampleblock}{题解}
考虑乘法, 只能是 $2\times 3 = 6$ 或 $2\times 4 = 8$. \\
若是 $2\times 3 = 6$, 还剩 $1,4,5,7,8,9$, 
组合成 $1 + 7 = 8,\quad 4 + 5 = 9$. \\
若是 $2\times 4 = 8$, 还剩 $1,3,5,6,7,9$, 组合不成.
\end{exampleblock}
\end{frame}

\begin{frame}{习题 A3}
满足下面算式的填法共有多少种?
$$6 + \square = \square \square$$
\begin{exampleblock}{题解}
枚举第一个 $\square$, 可以是 $4,5,6,7,8,9$.
\end{exampleblock}
\centering 
(习题 A4, A6, A7: 较难, 不讲.) \\
(习题 A5: 同例 4, 不讲.) \\
\end{frame}

\begin{frame}{习题 A8}
将 $1 \sim 9$ 分别填入下列算式的 $\square$ 内, \\
使每个等式都成立. (其中一个数字已填好)
$$
\begin{cases}
\square \times \square = 5\square \\
\square \square \div \square \times \square = \square \\
\end{cases}
$$
\begin{exampleblock}{题解}
第一个式子可以是 $6\times 9 = 54$ 或 $7\times 8 = 56$. \\
若 $6\times 9 = 54$, 还剩 $1,2,3,7,8$. 可以是
$12 \div 3,\quad$ 
$18 \div 2,\quad$ 
$18 \div 3,\quad$  
$21 \div 3,\quad$ 
$21 \div 7,\quad$ 
$27 \div 3,\quad$ 
$32 \div 8,\quad$ 
$72 \div 8,\quad$ \\
均不合题意. \\
若 $7\times 8 = 56$, 还剩 $1,2,3,4,9$. 可以是
$12 \div 3,\quad$ 
$12 \div 4,\quad$ 
$14 \div 2,\quad$   
$21 \div 3,\quad$  
$24 \div 3,\quad$  
$32 \div 4,\quad$ \\
只有 $12 \div 4 \times 3 = 9$ 符合题意.
\end{exampleblock}
\end{frame}

\begin{frame}{习题 A9}
将 $1 \sim 9$ 分别填入下列算式的 $\square$ 内, 
使每个等式都成立. 
$$
\begin{cases}
\square \times \square \times \square = \square + \square \\
\square \div \square = \square \div \square \\
\end{cases}
$$
\\[-1em]
\begin{exampleblock}{题解}
易知 $5,7$ 不能放在下面的式子中, \\
因此要么第一个式子右边是 $5 + 7$, 要么左边有 $5$ 或 $7$. \\
第一个式子左边可以是 
$\xcancel{1\times2\times3}$, 
$\xcancel{1\times2\times4}$, 
$1\times2\times5$, 
$1\times2\times6$, 
$1\times2\times7$, 
$\xcancel{1\times2\times8}$, 
$1\times3\times4$, 
$1\times3\times5$. \\
检验 $1\times 2\times 5 = 3 + 7$, 剩下 $4,6,8,9$ 不合题意. \\
检验 $1\times 2\times 6 = 5 + 7$, 剩下 $3,4,8,9$ 不合题意. \\
检验 $1\times 2\times 7 = 5 + 9$, 剩下 $3,4,6,8$ 可以. \\
检验 $1\times 3\times 4 = 5 + 7$, 剩下 $2,6,8,9$ 不合题意. \\
检验 $1\times 3\times 5 = 7 + 8$, 剩下 $2,4,6,9$ 不合题意. \\
\end{exampleblock}
\end{frame}

\begin{frame}{习题 A10}
将 $1 \sim 9$ 分别填入下列算式的 $\square$ 内, \\
使每个等式都成立. (其中一个数字已填好)
$$
\square \times \square = 
\square \square \square \div 5 \square =
\square \square
$$
\\[-1em]
\begin{exampleblock}{题解}
倒数第二个框必须是 $1$. 
枚举最后一个框, 可以是 $0,2,4,5,6,8$, \\
若是 $2\times 5=10$, 重复了 $5$. \\
若是 $2\times 6=12$, 重复了 $2$. \\
若是 $3\times 4=12$, 还剩 $6,7,8,9$. 尝试发现不合题意. \\
若是 $2\times 7=14$, 还剩 $3,6,8,9$. 尝试发现不合题意. \\
若是 $3\times 5=15$, 重复了 $5$. \\
若是 $2\times 8=16$, 还剩 $3,4,7,9$. 尝试发现不合题意. \\
若是 $2\times 9=18$, 还剩 $3,4,6,7$. 尝试发现不合题意. \\
若是 $3\times 6=18$, 还剩 $2,4,7,9$. 尝试发现 $18\times 54 = 972$. \\
\end{exampleblock}
\centering (习题 A11, A12, (B,C 卷): 较难, 不讲.) \\
\end{frame}
\end{document}
