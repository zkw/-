\documentclass[content.tex]{subfiles}

\title{小学数学思维潜力课 / 四年级暑假 \\ (习题课)}

\begin{document}

\begin{frame}
\maketitle
\end{frame}

\begin{frame}{习题 C1}
$
\begin{cases}
2A + B = 16 & (1)\\
A + 2B = 14 & (2)\\
\end{cases}
$, \quad 求 $A, B$? 
\begin{exampleblock}{题解}
$(1)\times2 - (2)$, $3A=18\implies A=6$ . \\
代入得 $B = 16-2A = 4$ .
\end{exampleblock}
\end{frame}

\begin{frame}{习题 C2}
$
\begin{cases}
3A + 2B = 34 & (1)\\
3A + 4B = 44 & (2)\\
\end{cases}
$, \quad 求 $A, B$? 
\begin{exampleblock}{题解}
$(2) - (1)$, $2B=10\implies B=5$ . \\
代入得 $A = (34-2B)/3 = 8$ .
\end{exampleblock}
\end{frame}

\begin{frame}{习题 C3}
$
\begin{cases}
2\text{梨} + \text{柿子} + \text{香蕉} = 17\\
\text{梨} + 2\text{柿子} + \text{香蕉} = 14\\
\text{梨} + \text{柿子} + 2\text{香蕉} = 13\\
\end{cases}
$, 求 $\text{梨}, \text{柿子}, \text{香蕉}$? 
\begin{exampleblock}{题解}
三式相加, $4\text{梨} + 4\text{柿子} + 4\text{香蕉} = 44$ . \\
$\implies \text{梨} + \text{柿子} + \text{香蕉} = 11$ . \\
分别用各式减去此式, $\text{梨} = 6$, $\text{柿子} = 3$, $\text{香蕉} = 2$.
\end{exampleblock}
\end{frame}

\begin{frame}{习题 C4}
在一次考试中, \\
小强和小玲的成绩加起来是 $199$ 分, \\
小玲和小芳的成绩加起来是 $198$ 分, \\
小强和小芳的成绩加起来是 $197$ 分, \\
问他们三人的成绩各是多少分?
\begin{exampleblock}{题解}
设小强 $x$ 分, 小玲 $y$ 分, 小芳 $z$ 分. \\
$
\begin{cases}
x + y = 199 \\
y + z = 198 \\
x + z = 197 \\
\end{cases}
\implies 2x + 2y + 2z = 594 \implies x + y + z = 297 
$ . 分别减去三个式子得到 $x = 99, y = 100, z = 98$ .
\end{exampleblock}
\end{frame}

\begin{frame}{习题 C5}
小红买 $1$ 件上衣和 $2$ 条裤子共用去 70 元, \\
又知道 $2$ 条裤子比 $1$ 件上衣贵 10 元, \\
问小红买的裤子每条多少钱?
\begin{exampleblock}{题解}
设上衣 $x$ 元, 裤子 $y$ 元. \\
$
\begin{cases}
x + 2y = 70 \\
2y - x = 10 \\
\end{cases}
\implies 4y = 80 \implies y = 20 
$ .
\end{exampleblock}
\end{frame}

\begin{frame}{习题 C6}
学期结束时, 小军, 小亮, 小强去称体重. \\
小军和小亮站在一起是重 $65$ 千克, \\
小亮和小强站在一起是重 $67$ 千克, \\
小军和小强站在一起是重 $66$ 千克, \\
问这三人的体重各是重多少千克?
\begin{exampleblock}{题解}
设小军 $x$ 千克, 小亮 $y$ 千克, 小强 $z$ 千克. \\
$
\begin{cases}
x + y = 65 \\
y + z = 67 \\
x + z = 66 \\
\end{cases}
\implies 2x + 2y + 2z = 198 \implies x + y + z = 99 
$ . 分别减去三个式子得到 $x = 32, y = 33, z = 34$ .
\end{exampleblock}
\end{frame}

\begin{frame}{习题 C7}
在文具店里, \\
$3$ 支钢笔和 $2$ 支铅笔的价钱一共 $17$ 元, \\
$2$ 支钢笔和 $2$ 支铅笔的价钱一共 $12$ 元, \\
问钢笔和铅笔的价钱各是多少元?
\begin{exampleblock}{题解}
设钢笔 $x$ 元, 铅笔 $y$ 元. \\
$
\begin{cases}
3x + 2y = 17 \\
2x + 2y = 12 \\
\end{cases}
\implies x = 5 \implies y = 1 
$ .
\end{exampleblock}
\end{frame}

\begin{frame}{习题 C8}
工人们加工一批零件. 已知\\
甲乙共加工 $77$ 个, \\
乙丙共加工 $69$ 个, \\
甲丙共加工 $56$ 个, \\
问甲乙丙各加工多少个零件?
\begin{exampleblock}{题解}
设甲 $x$ 个, 乙 $y$ 个, 丙 $z$ 个. \\
$
\begin{cases}
x + y = 77 \\
y + z = 69 \\
x + z = 56 \\
\end{cases}
\implies 2x + 2y + 2z = 202 \implies x + y + z = 101 
$ . 分别减去三个式子得到 $x = 32, y = 45, z = 24$ .
\end{exampleblock}
\end{frame}

\begin{frame}{习题 C9 (有修改)}
$3$ 头牛和 $8$ 只羊一天共吃青草 $93$ 千克, \\
$5$ 头牛和 $10$ 只羊一天共吃青草 $135$ 千克, \\
问 $1$ 头牛和 $1$ 只羊一天共吃青草多少千克?
\begin{exampleblock}{题解}
设牛一天吃 $x$ 千克, 羊一天吃 $y$ 千克. \\
$
\begin{cases}
3x + 8y = 93   & (1) \\
5x + 10y = 135 & (2) \\
\end{cases}, \quad
$ $(1) - (2) \div 5 \times 3$, \\
$2y=12\implies y = 6\implies x = 15\implies x + y = 21$ .
\end{exampleblock}
\end{frame}

\begin{frame}{习题 C10}
饲养场出售鸡鸭, 爸爸 \\
买 $2$ 只鸭和 $3$ 只鸡共用 $42$ 元, \\
买 $3$ 只鸭和 $2$ 只鸡共用 $43$ 元, \\
问一只鸭和一只鸡各多少元?
\begin{exampleblock}{题解}
设一只鸭 $x$ 元, 一只鸡 $y$ 元. \\
$
\begin{cases}
2x + 3y = 42   & (1) \\
3x + 2y = 43   & (2) \\
\end{cases}, \quad
$ $(2) \times 3 - (1) \times 2$, \\
$5x=45\implies x=9\implies y=(42-2x)/3=8$ .
\end{exampleblock}
\end{frame}

\begin{frame}{习题 C11}
有梨 $99$ 千克, 分给甲乙丙三组. \\
甲组比乙组多分 $4$ 千克, \\
乙组比丙组多分 $4$ 千克, \\
问三个组各得梨多少千克?
\begin{exampleblock}{题解}
设丙组 $x$ 千克, 则乙组 $x + 4$ 千克, 丙组 $x + 8$ 千克. \\
$x + (x + 4) + (x + 8) = 99 \implies x=29$.
\end{exampleblock}
\end{frame}

\begin{frame}{习题 C12}
百货商店中, \\
两只圆珠笔与三只蘸水笔共值 $7$ 角 $8$ 分, \\
三只圆珠笔与两只蘸水笔共值 $7$ 角 $2$ 分, \\
问一支圆珠笔值多少钱?
\begin{exampleblock}{题解}
设一只圆珠笔 $x$ 元, 一只蘸水笔 $y$ 元. \\
$
\begin{cases}
2x + 3y = 78   & (1) \\
3x + 2y = 72   & (2) \\
\end{cases}, \quad
$ $(2) \times 3 - (1) \times 2$, \\
$5x=60\implies x=12\implies y=(72-3x)/2=18$ .
\end{exampleblock}
\end{frame}

\begin{frame}{习题 C1}
求值: $454+220-154+80$
\begin{exampleblock}{题解}
交换次序, 添括号. $\text{原式} = (454-154)+(220+80) = 300+300=600$ .
\end{exampleblock}
\end{frame}

\begin{frame}{习题 C2}
求值: $86+57-17+114+234$
\begin{exampleblock}{题解}
交换次序, 添括号. \\
$\text{原式} = (86+114)+(57-17)+234 = 200+40+234=474$ .
\end{exampleblock}
\end{frame}

\begin{frame}{习题 C3}
求值: $9+98+997+1+2+3$
\begin{exampleblock}{题解}
交换次序, 添括号. \\
$\text{原式} = (9+1)+(98+2)+(997+3) = 10+100+1000=1110$ .
\end{exampleblock}
\end{frame}

\begin{frame}{习题 C4}
求值: $99-1+98-2+97-3+96-4$
\begin{exampleblock}{题解}
交换次序, 添括号. \\
$\text{原式} = (99-4)+(98-3)+(97-2)+(96-1) = 95\times 4=380$ .
\end{exampleblock}
\end{frame}

\begin{frame}{习题 C5}
求值: $67+360-60$
\begin{exampleblock}{题解}
添括号. $\text{原式} = 67+(360-60) = 367$ .
\end{exampleblock}
\end{frame}

\begin{frame}{习题 C6}
求值: $750-90+50$
\begin{exampleblock}{题解}
交换次序. $\text{原式} = 750+50-90 = 710$ .
\end{exampleblock}
\end{frame}

\begin{frame}{习题 C7}
求值: $43+260+37+140$
\begin{exampleblock}{题解}
交换次序, 添括号. $\text{原式} = (43+37) + (260+140) = 480$ .
\end{exampleblock}
\end{frame}

\begin{frame}{习题 C8}
求值: $162+88+112-62$
\begin{exampleblock}{题解}
交换次序, 添括号. $\text{原式} = (162-62) + (88+112) = 300$ .
\end{exampleblock}
\end{frame}

\begin{frame}{习题 C9}
求值: $95-130+105+530$
\begin{exampleblock}{题解}
交换次序, 添括号. $\text{原式} = (95+105) + (530-130) = 600$ .
\end{exampleblock}
\end{frame}

\begin{frame}{习题 C10}
求值: $238+59-38+41$
\begin{exampleblock}{题解}
交换次序, 添括号. $\text{原式} = (238-38) + (59+41) = 300$ .
\end{exampleblock}
\end{frame}

\begin{frame}{习题 C11}
求值: $75+538-338+25$
\begin{exampleblock}{题解}
交换次序, 添括号. $\text{原式} = (75+25) + (538-338) = 300$ .
\end{exampleblock}
\end{frame}

\begin{frame}{习题 C12}
求值: $456+97-104$
\begin{exampleblock}{题解}
添括号. $\text{原式} = 456+(97-104) = 456-7 = 449$ .
\end{exampleblock}
\end{frame}

\begin{frame}{习题 C13}
求值: $746-97+109$
\begin{exampleblock}{题解}
交换次序, 添括号. $\text{原式} = 746+(109-97) = 746+12 = 758$ .
\end{exampleblock}
\end{frame}

\begin{frame}{习题 C14}
求值: $6+8+97+997$
\begin{exampleblock}{题解}
叠加一个 $-6+0+3+3=0$ 和不变. \\
$\text{原式} = 0+8+100+1000 = 1108$ .
\end{exampleblock}
\end{frame}

\begin{frame}{习题 C15}
求值: $30+95+995+3995$
\begin{exampleblock}{题解}
叠加一个 $-15+5+5+5=0$ 和不变. \\
$\text{原式} = 15+100+1000+4000 = 5115$ .
\end{exampleblock}
\end{frame}

\begin{frame}{习题 C16}
求值: $6996+999+97+97$
\begin{exampleblock}{题解}
叠加一个 $4+1+3-8=0$ 和不变. \\
$\text{原式} = 7000+1000+100+89 = 8189$ .
\end{exampleblock}
\end{frame}

\begin{frame}{习题 C17}
求值: $9+19+199+1999$
\begin{exampleblock}{题解}
叠加一个 $-3+1+1+1=0$ 和不变. \\
$\text{原式} = 6+20+200+2000 = 2226$ .
\end{exampleblock}
\end{frame}

\begin{frame}{习题 C18}
求值: $99+98+97+96+95$
\begin{exampleblock}{题解}
等差数列, 平均数等于中间数. $\text{原式} = 97\times 5 = 485$ .
\end{exampleblock}
\end{frame}

\begin{frame}{习题 C19}
求值: $123+97+105-98-104$
\begin{exampleblock}{题解}
交换次序, 添括号. $\text{原式} = 123+(97-98)+(105-104) = 123$ .
\end{exampleblock}
\end{frame}

\begin{frame}{习题 C20}
求下表中所有数的和.
$$
\begin{matrix}
1 & 2 & 3 & 4 & 5 & 6 & 7 & 8 & 9 & 10 \\
2 & 3 & 4 & 5 & 6 & 7 & 8 & 9 & 10 & 11 \\
3 & 4 & 5 & 6 & 7 & 8 & 9 & 10 & 11 & 12 \\
4 & 5 & 6 & 7 & 8 & 9 & 10 & 11 & 12 & 13 \\
5 & 6 & 7 & 8 & 9 & 10 & 11 & 12 & 13 & 14 \\
6 & 7 & 8 & 9 & 10 & 11 & 12 & 13 & 14 & 15 \\
7 & 8 & 9 & 10 & 11 & 12 & 13 & 14 & 15 & 16 \\
8 & 9 & 10 & 11 & 12 & 13 & 14 & 15 & 16 & 17 \\
9 & 10 & 11 & 12 & 13 & 14 & 15 & 16 & 17 & 18 \\
10 & 11 & 12 & 13 & 14 & 15 & 16 & 17 & 18 & 19 \\
\end{matrix}
$$
\begin{exampleblock}{题解}
观察得到平均值是 $10$, 故 $10\times10\times10=1000$ .
\end{exampleblock}
\end{frame}

\end{document}
